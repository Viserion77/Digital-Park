\chapter{Introdução ao Documento}


O objetivo deste capítulo é apresentar o projeto. Para tal, deve-se desenvolver um texto, com as seguintes características: impessoalidade, objetividade, clareza, precisão, coerência e concisão. A introdução deve abrange os itens a seguir.

\section{Tema}

Neste item deve-se apresentar o tema do projeto, de forma clara e objetiva.


\section{Objetivo do Projeto}

Neste item devem ser descritos os objetos gerais e específicos do projeto como um todo. Independente do que será implementado, este item visa o entendimento global do projeto.

\section{Delimitação do Problema}


Neste item deve ser descrita a delimitação do problema, que define o ponto central do projeto. Isso quer dizer que, dentro de uma idéia geral do projeto, deve-se ressaltar a idéia específica efetivamente a ser desenvolvida. É neste item que a amplitude do projeto tem sua delimitação perfeitamente definida.

\section{Justificativa da Escolha do Tema}

Neste item deve-se expor a motivação acadêmica para a elaboração do projeto em questão, detalhando os motivos de ordem teórica ou de ordem prática para a sua realização.


\section{Método de Trabalho}

Neste item deve-se descrever o método a ser utilizado para realização do projeto, o tipo de processo de desenvolvimento de software1, a modelagem a ser utilizada (orientada a objeto, estruturada, outras).



\section{Organização do Trabalho}

Neste item deve-se descrever como o documento estará organizado.




\section{Glossário}
SCRUM:		Metodologia ágil de desenvolvimento de projetos.

MVP:		Produto com o mínimo valor possível, visado para validação da ideia do projeto.

API:		Interface para comunicação entre diferentes aplicações.

Feedback:	Retorno a um acontecimento.

Software:	Programa de computador.

UC:			Unidade Curricular.

